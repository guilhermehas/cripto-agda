\section{Introduction}

\subsection{History of Cryptocurrencies} 

In 1983, David Chaum created ecash \cite{panurach1996money} an anonymous cryptographic electronic money.
This cryptocurrency use \gls{rsa} blind signatures \cite{chaum1983blind} to spend transactions.
Later, in 1989, David Chaum found an electronic money corporation called DigiCash Inc.
It was declared bankruptcy in 1998.

Adam Back developed a \gls{pow} scheme for spam control, Hashcash \cite{back2002hashcash}.
To send an email, the hash of the content of this email plus a nonce has to have a numerical value
smaller than a defined target.
So, to create a valid email, the sender (miner) has to spend a considerable \gls{cpu} resource on it.
Miners should calculate a lot of hash values before finding a valid one.
This idea is used in Bitcoin proof of work because each block has a nonce guessed by the miner and
the hash of the block has to be less than the target value.

Wei Dai proposes b-money \cite{dai1998b} for the first proposal for distributed digital scarcity.
And Hal Finney created Bit Gold \cite{wallace2011rise}, a reusable proof of work for hash cash for
its algorithm of proof of work.

On 31 October 2008, Satoshi Nakamoto registered the website ``bitcoin.org'' and put a link for his
paper \cite{nakamoto2008bitcoin} in a cryptography mailing list.
In January 2009, Nakamoto released the Bitcoin software as an open-source code.
The identity of Satoshi Nakamoto is still unknown.
Since that time, the total market of Bitcoin came to 330 billion dollars in 17 of December of 2018
when its value reached a historic peak of 20 thousand dollars.

Other cryptocurrencies like Ethereum \cite{wood2014ethereum}, Monero \cite{noether2015ring} and
ZCash \cite{hopwood2016zcash} were created after Bitcoin,
but Bitcoin is still the cryptocurrency with the biggest market value.

Ethereum is a cryptocurrency that uses an account model instead of \gls{utxo} used in Bitcoin for its
transaction data structure.
It uses Solidity as its programming language for smart contracts which resembles Javascript,
so it is easier to program in it than in the stack machine programming language of Bitcoin.
Ethereum changed from proof of work (used in Bitcoin) to proof of stake
which is now the default proof mechanism of Ethereum 2.0.

Monero and ZCash are both cryptocurrencies that focus on fungibility, privacy, and decentralization.
Monero uses an obfuscated public ledger, so anyone can send transactions,
but nobody can tell the source, amount or destination.
ZCash uses the concept of zero-knowledge proof called \gls{kzsnark},
which guarantees privacy for its users.

\subsection{Objectives}

Cryptocurrencies are used as money and in smart contracts in a decentralized way.
Because of that, it is not possible to revert a transaction or undo the creation of the smart contract.
There is no legal framework or agent to solve a problem in case of the existence of a bug.
Because of that, the formal proofs are desirable in the cryptocurrency protocol.
So it can avoid big financial loss.

In the case of Bitcoin, if there is some problem in the source code,
it is possible to fix it using soft or hard forks.
In soft fork, there is an upgrade in the software that is compatible with the old software.
So it is possible the existence of old and new nodes in the same Bitcoin network.
In hard forks, all the nodes should be upgraded at the same time.
Because the newer version is not compatible with the older one.
So it is very dangerous to do this kind of fork.
Therefore in Bitcoin, this kind of fork just happened twice by accident.
It happened in 2013, because of the BerkeleyDB issue and it was solved in BIP 0050.
And happened in 2018, because of Denial of Service and inflation vulnerability bug.

In Bitcoin, the uniqueness of transaction \gls{id} was not guaranteed.
To fix this problem, it should put the block number in the coinbase transaction.
This kind of change was solved in a soft fork named SegWit.

In Ethereum, there was a bug in \gls{dao} smart contract.
Malicious users exploited a vulnerability in it with a
total loss of 150 million dollars.
There was a hard fork to undo most of the transactions that exploited this contract.
This kind of hard fork violates the principle that smart contracts should be ruled just by
algorithms without any human intervention.
The old Ethereum blockchain that has not done the fork became the Ethereum Classic.
It is the version of Ethereum that has never done a hard fork before.

The objective of this work is to give a formal definition of what a cryptocurrency should be.
There are some different definitions of a cryptocurrency in this work,
but there are some formal proofs that they are the same.

Given that cryptocurrencies and smart contracts should have guarantees of their correctness,
it is very important to verify their protocols.
The creation of formal methods for Bitcoin protocol and the proof of its correctness are
the objective of this work.

So, we are proposing a model of Bitcoin using an interactive theorem prover and dependently
typed programming language Agda.
We include a formalization of transactions, transactions tree, blocks, blockchain and
how messages are signed.
In this work, we also include functions that transform raw transactions (made with simple types like
lists, naturals) to transactions with dependent types.

\subsection{Related Work}

Beukema \cite{beukema2014formalising} was one of the first to try
to define a formal specification of Bitcoin.
In this works, functions interfaces of Bitcoin and what they should do were defined.
Most of these functions define how the Bitcoin Network protocol should be.
In this work, he does not utilize any programming language with dependent types like Agda or CoQ.
mCRL2, a specification programming language, was used.

Chaudhary and his team \cite{chaudhary2015modeling} have created a model of Bitcoin blockchain
in the model checker UPPAAL.
In his work, they calculate the probability of a malicious attack to succeed in doing a double spend.
For a small number of blocks, it is easier to do this attack.
Because of that, it is usually recommended that the user waits for more blocks confirmations
after a big transaction.

Bastiaan \cite{bastiaan2015preventing} showed a stochastic model of Bitcoin using
continuous Markov chains.
In his work, he proposes a way of avoiding a 51\% attack in the network,
using two-phase proof of work.

Orestis Melkonian \cite{melkonian2019formalizing} in his masters have done the formal specification
of BitML (smart contract language) in Agda.
BitML can be compiled to Script, the smart contract language of Bitcoin. 

Kosba \cite{kosba2016hawk} in his work made a programming language called Hawk for smart contracts.
This language uses formal methods to verify privacy using zero-knowledge proofs.
Using this language, the programmer does not have to worry about implementing the cryptography,
because the compiler generates automatically an effective one.

Bhargavan \cite{bhargavan2016formal} translated Solidity and Ethereum bytecode into F*.
He verified that the Ethereum \gls{dao} bug was caught in its translation.
Nowadays, they have an implementation of \gls{evm} and Solidity in OCaml,
but they want to have a full implementation of \gls{evm} in F* too.

Luu \cite{luu2016making} built a symbolic execution tool named OYENTE to look for potential bugs.
In his work, he found a lot of contracts with real bugs.
One of these bugs was the \gls{dao} bug, that caused a loss of 60 million dollars.
He used Z3 to find a potentially dangerous path of code.

Anton Setzer \cite{setzer2018modelling} also contributed to modeling Bitcoin.
He coded in Agda the definitions of transactions and
transactions tree of Bitcoin.
Orestis Melkonian starts to formalize Bitcoin Script.

My work tries to extend the Anton Setzer model and makes it possible to use the Bitcoin protocol
from inputs and outputs from plain text.
For example, the user sends a transaction in plain text to the software and it validates if it is correct.
To use the Anton Setzer model, the user has to send the data and the proof that are both valid.
