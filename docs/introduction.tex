\section{Introduction}

\subsection{Context} 

In 1983, David Chaum created ecash \cite{panurach1996money} an anonymous cryptographic eletronic money.
This cryptocurrency use \gls{rsa} blind signatures \cite{chaum1983blind} to spend transactions.
Later, in 1989, David Chaum found an electronic money corporation called DigiCash Inc.
It was declared bankruptcy in 1998.

Adam Back developed a \gls{pow} scheme for spam control, Hashcash \cite{back2002hashcash}.
To send an email, the hash of the content of this email plus a nonce has to have a numerical value
smaller than a defined target.
So, to create a valid email, the sender (miner) has to spend a considerable \gls{cpu} resource on it.
Because hash functions produce practically random values, so the miner has to guess a lot of nonce
values before finding some nonce that makes the hash of the email less than the target value.
This idea is used in Bitcoin proof of work because each block has a nonce guessed by the miner and
the hash of the block has to be less than the target value.

Wei Dai propose b-money \cite{dai1998b} for the first proposal for distributed digital scarcity.
And Hal Finney created Bit Gold \cite{wallace2011rise}, a reusable proof of work for hash cash for
its algorithm of proof of work.

On 31 October 2008, Satoshi Nakamoto registered the website ``bitcoin.org'' and put a link for his
paper \cite{nakamoto2008bitcoin} in a cryptography mailing list.
In January 2009, Nakamoto released the Bitcoin software as an open-source code.
The identity of Satoshi Nakamoto is still unknown.
Since that time, the total market of Bitcoin came to 330 billion dollars in 17 of December of 2018
when its value reached a historic peak of 20 thousand dollars.

Other cryptocurrencies like Ethereum \cite{wood2014ethereum}, Monero \cite{noether2015ring} and
ZCash \cite{hopwood2016zcash} were created after Bitcoin,
but Bitcoin is still the cryptocurrency with the biggest market value.

Ethereum is a cryptocurrency that uses an account model instead of \gls{utxo} used in Bitcoin for its
transaction data structure.
It uses Solidity as its programming language for smart contracts which resembles Javascript,
so it is easier to program in it than in the stack machine programming language of Bitcoin.
Ethereum is now transitioning from proof of work (used in Bitcoin) to proof of stake
which will be the default proof mechanism of Ethereum 2.0 and will be released in
3 of January of 2020.

Monero and ZCash are both cryptocurrencies that focus on fungibility, privacy and decentralization.
Monero uses an obfuscated public ledger, so anyone can send transactions,
but nobody can tell the source, amount or destination.
Zcash uses the concept of zero-knowledge proof called \gls{kzsnark},
which guarantee privacy for its users.

\section{Objectives}

\subsection{History}

Cryptocurrencies are used as money and used in smart contracts in a decentralized way.
Because of that, it is not possible to revert a transaction or undo the creation of the smart contract.
There is no legal framework or agent to solve a problem in case of the existence of a bug.
Because of that, the formal proofs are necessary in the cryptocurrency protocol.
So it can avoid big financial loss.

In the case of Bitcoin, if there is some problem in the source code,
it is possible to fix it using soft or hard forks.
In soft fork, there is an upgrade in the software that is compatible with the old software.
So it is possible the existence of old and new nodes in the same Bitcoin network.
In hard forks, all the nodes should be upgraded at the same time.
Because the newer version is not compatible with the older one.
So it is very dangerous to do this kind of fork.
Therefore in Bitcoin, this kind of fork just happened twice by accident.
It happened in 2013, because of BerkeleyDB issue and it was solved in BIP 0050.
And happend in 2018, becase of Denial of Service and inflation vulnerability bug.

For example, in Bitcoin, the uniqueness of transaction \gls{id} were not guaranteed.
To fix this problem, it should put the block number in the coinbase transaction.
This kind of change was solved in a soft fork named SegWit.

In Ethereum, there was a bug in \gls{dao} smart contract.
Because of that, malicious users exploited a vulnerability in it.
The total loss of this exploit was 150 million dollars on this day.
There was a hard fork to undo most of the transactions that exploited this contract.
This kind of hard fork violates the principle that smart contracts should be ruled just by
algorithms without any human intervention.
Because of that, the Ethereum blockchain that has not done the fork becomes the Ethereum classic.
It is the version of Ethereum that has never done a hard fork before.

\subsection{Proposes}

The objective of this work is to give a formal definition of what a cryptocurrency should be.
There are some different definitions of a cryptocurrency in this work,
but there are some formal proofs that they are the same.

In this work, it is possible to generate proofs transactions from transactions without proofs.
This means that a user can send a simple transaction without he worried to have to prove that
the transaction is right to put in the blockchain.
In Bitcoin, it happened in the same way.
Because the node has to verify the transactions.
