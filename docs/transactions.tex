\subsection{Transactions}

\subsubsection{Definitions}

In Bitcoin, there are some transactions.
In each transaction, there are multiple inputs and outputs.
Each input is named TXFieldWithId.
The input of one transaction is the output of another transaction.
Firsts outputs are generated from coinbase transaction (created by the miner).
Each block has just one of this transaction.

\agda{Transactions}{VectorOutput}

Vector output is the vector of outputs transactions.
It is a non-empty vector.
In its representation, it is possible to know in what time it was created (time is the position of
they in all transactions),
what is his size (quantity of outputs fields)
and the total amount spent in this transaction,

\emph{elStart} is a proof that the position of TXFieldWithId is the last one,
because its position in the vector is the same as the last position (size) of the vector.
It is used after to specify which input is in the transaction.

\agda{Transactions}{TXSigned}

A signed transaction is composed of a non-empty list of inputs and outputs.
For each input, there is a signature that confirms that he accepted every output in the list of outputs.
And in the transaction, there is proof ($in \geq out$) that the total amount of money
in all inputs are bigger than the total amount of outputs.
The remainder will be used by the miner.

\subsubsection{Raw Transaction}

Raw transactions are transactions without any explicit dependent type.
Here the definition of \emph{raw signed transaction}:

\agda{RawTransactions}{rawtxsigned}

\emph{Raw signed transactions} is a record with \emph{inputs}, \emph{outputs}
and the signature of \emph{inputs} and \emph{outputs}.

The definition of some important types:

\agda{Crypto}{cryptoTypes}

The definition of \emph{Raw Input}:

\agda{RawTransactions}{rawinput}

In each input, it is necessary to know the time, the position of it in the transaction,
the amount spent, its message, the signature, and its public key.
The signature is the signature of the message.
And the message is usually related to the amount spent in each output.

The definition of \emph{raw transaction}:

\agda{RawTransactions}{rawTransaction}

It is all inputs and all outputs.

The definition of \emph{Raw TX}:

\agda{RawTransactions}{rawTX}

The definition of \emph{raw transaction coinbase}:

\agda{RawTransactions}{rawCoinbase}

The defintion of \emph{raw Vector Output}: 

\agda{RawTransactions}{rawVecOut}

It has the time, its size, the total amount, the \emph{vector output}
and proof that this vector is the same as the list of outputs of this type.

The definition of the record that every input transaction is signed in a given time:

\agda{RawTransactions}{txsigall}

It has the size of vector output, the sublist of all inputs, the total amount,
the \emph{vector output} and a proof that all sublist of inputs are signed.

$rawTXSigned \to TXSigAll$ returns a signed transaction of all inputs in a given time
if \emph{rawTXSigned} has valid signatures for all these inputs:

\agda{RawTransactions}{rawtxSigToTxsigAll}

It has to validate first that the \emph{list of outputs} is a valid \emph{Vector Output}.
Second, it validates if the signature of the inputs are valid with the
\emph{raw signed transaction}.
In the last case, it validates if the time of the \emph{vector output} is equal
of the time of this transaction.
If all conditions match, it returns a proven signed transaction.
If not, it returns nothing.

\hyperref[rawToTX]{This function} transforms a \emph{raw transaction} into a \emph{signed transaction}:

\plabel{rawToTX}
\agda{RawTransactions}{txrawToTxsig}

The function $vecOut \equiv ListAmount$ returns a proof that the \emph{vector output} is equal
to the total amount of the \emph{list of transactions}.
It is impossible that the \emph{vector output} is equal to an empty list.
In case that the list has just one element, it just has to return \emph{refl}.
The another case, it is done recursively.

The proof that the amount of input transaction is greater than the amount of output
is just a rewrite from the previous proof ($vecOut≡ListAmount outputs vecOut out≡vec$).

The function \emph{sameMessage} returns a proof that the message of \emph{raw transaction}
is the same as the message of the \emph{vector output}.
In case that \emph{vector output} has just size one or two, it is a trivial case.
The other cases are doing it recursively.

\emph{sigPub} is another function that returns a proof that an input message is signed.
It validates it with its public key.

The last function returns a proof that every input was signed.
It is done in a recursive way using the function \emph{sigPub}.

This is the function that transforms a list ot transactions into a possible \emph{vector output}:

\agda{RawTransactions}{listTXFieldtoVecOut}

The list has to be at least with a size one.
Because the \emph{vector output} can not be empty.
To add one element into the vector, it has to verify if the time is equal to the first time.
Another verification is that the informed position in the vector is right.
If all validations are right, it returns the vector output.
If it is not, it returns nothing.

The definition of the function that transform a \emph{raw transaction} into a
\emph{raw signed transaction}:

\agda{RawTransactions}{rawtoTXSigned}

The first validation that the function does is verifying that the outputs are not empty.
Another validation is verifying if the amount spent on inputs is greater than the amount of the outputs.
The function \emph{Signed?}, defined in the crypto library, validates if the
message was signed with the input.
After, it validates if all inputs are signed.
If all validations are right, it returns the \emph{raw transaction signed}.
If it is not, it returns nothing.
