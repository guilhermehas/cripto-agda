\subsection{Crypto Functions}
The first thing that we define is the crypto functions that will be needed to make the cryptocurrency.
Messages can be defined in multiple ways, one array of bytes, one string or a natural number.
Messages in this context means some data.

The private key is a number, a secret that someone has.
In Bitcoin, the private key is a 256-bit number.
A private key is used to signed messages.

The public key is generated from a private key.
But getting the private key from a public key is really difficult.
To verify who signed a message with a private key, he has to show the public key.

Hash is an injection function (the probability of two functions having the same hash is very low).
The function is used from a big domain to a small domain.
For example, a hash of big file (some GBs) is an integer of just some bytes.
It is very useful to prove for example that 2 files are equal.
If the hash of two files are equal, the probability of these files being equal is really high.
It is used in torrents clients, so it is safe to download a program to untrusted peers,
just have to verify if the hash of the file is equal to the hash of the file wanted.

These functions can be defined, but it is not the purpose of this thesis.
So they will be just postulates.

\agda{Crypto}{cryptoPostulates}
