\section{Relevant Background}

\subsection{Literature Review}

Before this work, there was some research in this field.
Beukema \cite{beukema2014formalising} was one of the first to try
to define a formal specification of Bitcoin.
In this work, he defines functions interfaces of Bitcoin and what they should do.
Most of these functions define how the Bitcoin Network protocol should be.
In his work, he does not utilize any programming language with dependent types like Agda or CoQ.
He uses mCRL2, a specification programming language.

Chaudhary \cite{chaudhary2015modeling} and his team have created a model of Bitcoin blockchain
in the model checker UPPAAL.
In his work, he calculates the probability of a malicious attack to succeed in doing a double spend.
For a small number of blocks, it is easier to do this attack.
Because of that, it is usually recommended that the user wait more blocks confirmations
after a big transaction.

Bastiaan \cite{bastiaan2015preventing} showed a stochastic model of Bitcoin using
continuous Markov chains.
In his work, he proposes a way of avoiding a 51\% attack in the network,
using two-phase proof of work.

Orestis Melkonian \cite{melkonian2019formalizing} in his masters have done the formal specification
of BitML (smart contract language) in Agda.
This language can be compiled to Script, the smart contract language of Bitcoin. 

Kosba \cite{kosba2016hawk} in his work made a programming language called Hawk for smart contracts.
This language uses formal methods to verify privacy using zero-knowledge proofs.
Using this language, the programmer does not have to worry about implementing the cryptography,
because the compiler generates automatically an efficient one.

Bhargavan \cite{bhargavan2016formal} translated Solidity and Ethereum bytecode into F*.
He verified that the Ethereum DAO bug was caught in its translation.
Nowadays, they have an implementation of Ethereum Virtual Machine (EVM) and Solidity in OCaml,
but they want to have a full implementation of EVM in F* too.

Luu \cite{luu2016making} built a symbolic execution tool named OYENTE to look for potential bugs.
In his work, he found a lot of contracts with real bugs.
One of these bugs was TheDAO bug, that caused a loss of 60 million dollars.
He used Z3 to find a potentially dangerous path of code.

Anton Setzer \cite{setzer2018modelling} also contributed to modeling Bitcoin.
He coded in Agda the definitions of transactions and
transactions tree of Bitcoin.
Orestis Melkonian start to formalize Bitcoin Script.

My work tries to extend Anton Setzer model and makes it possible to use the Bitcoin protocol
from inputs and outputs from plain text.
For example, the user sends a transaction in plain text to the software and it validates if it is correct.
To use the Anton Setzer model, the user has to send the data and the proof that are both valid.

\subsection{Agda Introduction}
Agda is a dependently typed functional language developed by Norell at Chalmers University of Technology
as his Ph.D. Thesis.
The current version of Agda is Agda 2.

  \subsubsection{Syntax}
  In Agda, \emph{Set} is equal to type.
  In languages with dependent types, it is possible to create a function that returns a type.

  \agda{agdaExamples}{funcType}

  After the function name, it is two colon (\emph{:}) and the arguments of the function.
  It is closed by \emph{(name\_of\_argument : type\_of\_argument)}.
  After all, there is one arrow and the type of the result of the function.
  This ``if, then, else'' is not a function built-in in Agda.
  It is a function defined this way \emph{if\_then\_else\_} .

  So it is possible to use this function in the default way.

  \agda{agdaExamples}{funcTypeUnd}

  Or use the arguments inside the underscore.

  \agda{agdaExamples}{funcType2}

  The same notation can be done using just arrows without naming the arguments.

  Because of dependent types, it is possible to have a type that depends on the input.

  \agda{agdaExamples}{dependentType}

  It is possible in Agda to do pattern match.
  So it breaks the input in cases.

  \agda{agdaExamples}{patternMatch}

  To create a new type with a different pattern match, it is used the data constructor.

  \agda{agdaExamples}{dataConstructor}

  This is another example of \emph{Data Set}, but it depends on the argument.

  \agda{agdaExamples}{vector}

  \emph{Vector zero} is a type of a vector of size zero, so the only option to construct it is the empty vector.
  It is constructed from the first constructor.
  Other types of vectors like \emph{Vector 1} (vector of size one), \emph{Vector 2}, ... can only be constructed by
  the second constructor.
  It takes as argument a natural number and a vector and returns a vector with the size of the last vector
  plus one.

  Records are data types with just one case of pattern match.

  \agda{agdaExamples}{record}

  The constructor is the name of the data constructor.

  Implicits terms are elements that the compiler is smart enough to deduce it.
  So it is not necessary to put it as an argument of the function.

  \agda{agdaExamples}{id}

  Implicits arguments are inside \emph{\{\}}.
  In this example, the name of the Set (\emph{A}) can not be omitted
  (like the second function version of boolean to set),
  because it is used to say that \emph{x} is of type \emph{A}.

  In the case of the function \emph{id}, the type of input can be deduced by the compiler.
  For example, the only type that \emph{zero} can be is Natural.

  \agda{agdaExamples}{idNat}

  Functions in Agda can be defined in two ways

  \agda{agdaExamples}{funcs}

  In the first case, the arguments are before equal sign (\emph{=}).
  In the second way, it is used the lambda abstraction that means the same thing.

  \subsubsection{Lambda Calculus}
  Lambda Calculus is a minimalist Turing complete programming language with the concept of abstraction,
  application using binding and substitution. For example, \emph{x} is a variable, $(\lambda x.M)$
  is an Abstraction and (\emph{M N}) is an Application.

  In Lambda Calculus, there are two types of conversions $\alpha$-conversion and $\beta$-reduction.
  In $\alpha$-conversion, $(\lambda x.M[x]) \rightarrow (\lambda y.M[y])$.
  So in every free variable in \emph{M} will be renamed from \emph{x} to \emph{y}.
  For \emph{M[x] = x}, an $\alpha$-conversion is $(\lambda x.x) \rightarrow (\lambda y.y)$

  A free variable is every variable that is not bound outside.
  For example, $((\lambda\textcolor{green}{x}.\textcolor{blue}{x}) \textcolor{red}{x})$.
  The \textcolor{blue}{blue x} is binded for the \textcolor{green}{green x},
  but the \textcolor{red}{red x}
  is not binded for any function. So the \textcolor{red}{red x} is a free variable.

  In $\beta$-reduction, it replaces the all free for the expression in the application.
  The $\beta$-reduction of this expression $((\lambda x.M) N) \rightarrow (M[x := N])$ .
  So if $M = x$, the $\beta$-reduction will be $((\lambda x.x) N) \rightarrow N$.
  If $M = (\lambda x.x) x$, the $\beta$-reduction will be
  $(\lambda x.((\lambda x.x)x))N \rightarrow (\lambda x.x)N$.

  Agda uses typed lambda calculus.
  So in an application \emph{(M N)}, \emph{M} has to be of type $A \Rightarrow B$ and N has to be of type A.
  $(\lambda (x : A) . x)$ is of type $A \Rightarrow A$, because \emph{x} is of type \emph{A}.

  \agda{lambdaCalculus}{Id}
  The simplest function is the identity function made in Agda.

  \agda{lambdaCalculus}{Id2}
  This is another way of writing the same function.

  \agda{lambdaCalculus}{trueFalse}
  This is how true and false are encoded in lambda calculus.

  \agda{lambdaCalculus}{naturals}
  This is how naturals numbers are defined in lambda calculus.
  Look that the definition of zero looks like the definition of false.

  \agda{lambdaCalculus}{isZero}
  Defining natural numbers in this way, it is possible to say if a natural number is zero or not.

  \agda{lambdaCalculus}{plus}
  Plus is defined this way using lambda calculus.

  \agda{lambdaCalculus}{onePone}
  This is one example of the calculation of one plus one in Lambda Calculus.

  \agda{lambdaCalculus}{list}
  This is how lists are defined in Lambda Calculus.

  \agda{lambdaCalculus}{sumList}
  Substituting the cons operation of list per plus and nil list to zero, it is possible to calculate
  the sum of the list.

  \agda{lambdaCalculus}{either}
  In this way, it is possible to define \emph{Either}.
  It is one way to create a type that can be a Natural or a Boolean.

  \agda{lambdaCalculus}{eitherExamples}
  In these examples, it is defined zero, one in left and false, true in right.

  \agda{lambdaCalculus}{eitherRes}
  \emph{Either} is useful when defining one function that works for left and another that works for the right.
  The function chosen for left was if a natural number is zero and
  the function chosen for right was if the identity function.

  \agda{lambdaCalculus}{tuple}
  This way is how tuple is defined in Lambda Calculus.

  \agda{lambdaCalculus}{tupleExamples}
  This is how is defined the tuple zero false and the tuple one true.

  \agda{lambdaCalculus}{tupleAdd}
  This is one way of defining a function that adds one to the argument
  if the first element of the tuple is true.


  \subsubsection{Martin-Löf type theory}
  Agda also provides proof assistants based on the intentional Martin-Löf type theory.

    In Martin-Löf type theory, there are 3 finite types and 5 constructors types.
    The 0 type contain 0 terms, it is called empty type and it is written bot.
    \agda{agdaExamples}{botType}

    The 1 type is the type with just 1 canonical term and it represents existence.
    It is called unit type and it is written top.
    \agda{agdaExamples}{trivialType}

    The 2 type contains 2 canonical terms. It represents a choice between two values.
    \agda{agdaExamples}{eitherType}

    The Boolean type is defined using the Trivial type and the Either type.
    \agda{agdaExamples}{boolType}

    If statement is defined using booleans.
    \agda{agdaExamples}{ifThenElse}


    \subsubsection{Types Constructors}
    The sum-types contain an ordered pair.
    The second type can depend on the first type.
    It has the same meaning of exist.
    \agda{agdaExamples}{sumType}

    The $\pi$-types contain functions.
    So given an input type, it will return an output type.
    It has the same meaning as a function:
    \agda{agdaExamples}{piType}

    In Inductive types, it is a self-referential type.
    Naturals numbers are examples of that:
    \agda{agdaExamples}{Nat}
    Other data structs like linked list of natural numbers, trees, graphs are inductive types too.

    Proofs in inductive types are made by induction.
    \agda{agdaExamples}{NatElim}

    Universe types are created to allow proofs written in all types.
    For example, the type of \emph{Nat} is \emph{U0}.


It looks like CoQ, but does not have tactics.
Agda is a total language, so it is guaranteed that the code always terminal and coverage all inputs.
Agda needs it to be a consistent language.

Agda has inductive data types that are similar to algebraic data types in non-dependently
typed programming language.
The definition of Peano numbers in Agda is:

\agda{agdaExamples}{Nat}

Definitions in Agda are done using induction.
For example, the sum of two numbers in Agda:

\agda{agdaExamples}{sum}

In Agda, because of dependent types, it is possible to make some restrictions in types that are not
possible in other languages.
For example, get the first element of a vector.
For it, it is necessary to specify in the type that the vector should have at size greater or equal
then that one.

\agda{agdaExamples}{vecHead}

Another good example is that in the sum of two matrices, they should have the same dimensions.

\agda{agdaExamples}{matrixSum}


\subsection{Bitcoin}

The Bitcoin was made to be a peer to peer electronic cash.
It was made in one way that users can save and verify transactions without the need of a trusted party.
Because of that no authority or government can block the Bitcoin.

\incimg{transactions1}{png}{Transaction}
%% https://medium.com/coinmonks/bitcoin-transactions-be401b48afe6

Transactions in Bitcoins (like in \ref{fig:transactions1}) are an array of input of
previous transactions and an array of outputs.
Each input and output is an address, each address is made from a public key
that is made from a private key.

\incimg{privatekey}{png}{Bitcoin account}
%% https://coinsutra.com/bitcoin-private-key/

A private key is a big number.
It is so big that it is almost impossible to generate two identicals private keys.

The public key is generated from the private key
(like in \ref{fig:privatekey} where account number is f(p)),
but a private key can not be generated from a public key.

The mining transaction does require an input.
For each input of the transaction, it is necessary a signature signed with a private key
(like in FIG2 where signature is f(p,t))
to prove the ownership of the Bitcoins.
With the message and the signature, it is possible to know that the owner of the private key
that generates the public key signed this message.

With the signature and the public key, it is not possible to know the private key.
In FIG2, the checker is a f(t,s,a).
So because of that, the owner of the private key can sign several messages without anyone knows
his private key.

\incimg{transactions2}{jpg}{Verification and signature of transactions}
%% https://en.wikipedia.org/wiki/Bitcoin_network

Transactions (shown in \ref{fig:transactions2}) are grouped in a block (shown in \ref{fig:blockchain}).
Each block contains in its header the timestamp of its creation, the hash of the block,
the previous hash and a nonce.
A nonce is an arbitrary value that the miner has to choose to make the hash of the block respect some
specific characteristics.

\incimg{blockchain}{png}{Blockchain}
%% https://bitcoin.stackexchange.com/questions/12427/can-someone-explain-how-the-bitcoin-blockchain-works/13347
%% https://i.stack.imgur.com/HrKX0.png

Each block has a size limit of 1 MB.
Because of that, Bitcoin forms a blockchain (a chain of blocks).
Each block should be created in an average of 10 minutes.
This time was chosen because 10 minutes is enough to propagate the block throughout the world.
To make the blockchain tamper-proof, there is a concept called proof of work in Bitcoin.
So the miner has chosen a random value as nonce that makes the hash of the block less
then a certain value.
This value is chosen in a way that each block should be generated on 10 minutes in average.
If the value is too low, miners will take more time to find a nonce that make the hash block
less than it.
If it is too high, it will be easier to find a nonce and they will find it faster.

When two blocks are mined in nearly the same time, there are two valid blockchains.
It is because the last block in both blockchains are valid but different.
Because of this problem, in the Bitcoin protocol, the largest chain is always the right chain.
While two valid chains have the same size, it is not possible to know which chain is the right.
This situation is called fork and when it happens, it is necessary to wait to see in which chain
the new block will be.

In Bitcoin, there is a possibility of a 51\% attack.
It happens when some miner, with more power than all network, mine secretly the blocks.
So if the main network has 50 blocks, the miner could produce hidden blocks from 46 to 55
and he would have 10 hidden blocks from the network.
When he shows their hidden blocks, his chain become the valid chain, because it is bigger.
So all transactions from previous blockchain from 46 to 50 blocks become invalid.
Because of that, when someone makes a big transaction in the blockchain, it is a good idea
to wait more time.
So it is becoming harder and harder to make a 51\% with more time.
Bitcoin has the highest market value nowadays, so attacking the Bitcoin network is very expensive.
Nowadays, this kind of attack is more common in new altcoins.

\incimgdiv{wallet}{png}{Wallet}{2}
%% https://trustwallet.com/buy-bitcoin/

Wallet (shown in \ref{fig:wallet}) is software that tracks all transactions 
that the users received and sent.
It also makes new transactions from previously received transactions.

\subsection{Ethereum}

Ethereum differs from Bitcoin in having an Ethereum Virtual Machine (EVM) to run script code.
EVM is a stack machine and Turing complete while Bitcoin Script is not
(it is impossible to do loops and recursion in Bitcoin).

Transactions in Bitcoin are all stored in the blockchain.
In Ethereum, just the hash of it is stored in it.
So it is saved in the off-chain database.
Because of that, it is possible to save more information in Ethereum Blockchain.

In Bitcoin, the creator of the contract to pay the amount proportional to its size.
In Ethereum, it is different, there is a concept of gas.
Each smart contract in Ethereum is made by a series of instructions.
Each instruction consumes different computational effort.
Because of that, in Ethereum, there is a concept of gas, that measure how much computational effort
each instruction needs.
So in each smart contract, it is well know how much computational effort will be necessary to run it
and it is measured in gas.
Because computational effort is a scarce resource, to execute the smart contract, it is necessary to
pay an amount in Ether for each gas to the miner run it.
Smart contracts that pay more ether per gas run first because the miner will want to have the best
profit and they will pick them.
If the amount of ether per gas paid is not high enough, the contract will not be executed,
because some other contracts pay more that will be executed instead of this one.

Because Ethereum has its EVM with more instructions than Bitcoin and it is Turing Complete,
it is considered less secure.
Ethereum has its high-level programming language called Solidity that looks like Javascript.
