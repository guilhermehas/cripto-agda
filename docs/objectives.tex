\section{Objectives}

\subsection{History}

Cryptocurrencies are used as money and used in smart contracts in a decentralized way.
Because of that, it is not possible to revert a transaction or undo the creation of the smart contract.
There is no legal framework or agent to solve a problem in case of the existence of a bug.
Because of that, the formal proofs are necessary in the cryptocurrency protocol.
So it can avoid big financial loss.

In the case of Bitcoin, if there is some problem in the source code,
it is possible to fix it using soft or hard forks.
In soft fork, there is an upgrade in the software that is compatible with the old software.
So it is possible the existence of old and new nodes in the same Bitcoin network.
In hard forks, all the nodes should be upgraded at the same time.
Because the newer version is not compatible with the older one.
So it is very dangerous to do this kind of fork.
Therefore in Bitcoin, this kind of fork just happened twice by accident.
It happened in 2013, because of BerkeleyDB issue and it was solved in BIP 0050.
And happend in 2018, becase of Denial of Service and inflation vulnerability bug.

For example, in Bitcoin, the uniqueness of transaction \gls{id} were not guaranteed.
To fix this problem, it should put the block number in the coinbase transaction.
This kind of change was solved in a soft fork named SegWit.

In Ethereum, there was a bug in \gls{dao} smart contract.
Because of that, malicious users exploited a vulnerability in it.
The total loss of this exploit was 150 million dollars on this day.
There was a hard fork to undo most of the transactions that exploited this contract.
This kind of hard fork violates the principle that smart contracts should be ruled just by
algorithms without any human intervention.
Because of that, the Ethereum blockchain that has not done the fork becomes the Ethereum classic.
It is the version of Ethereum that has never done a hard fork before.

\subsection{Proposes}

The objective of this work is to give a formal definition of what a cryptocurrency should be.
There are some different definitions of a cryptocurrency in this work,
but there are some formal proofs that they are the same.

In this work, it is possible to generate proofs transactions from transactions without proofs.
This means that a user can send a simple transaction without he worried to have to prove that
the transaction is right to put in the blockchain.
In Bitcoin, it happened in the same way.
Because the node has to verify the transactions.
