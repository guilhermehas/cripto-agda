\subsection{Transaction Tree}

\subsubsection{Definition}

The transaction tree is one of the most important data structures in Bitcoin.
In the transaction tree, there are all unspent transaction outputs (UTXO).
In every new transaction, the UTXOs used as input is removed from the transaction tree.

\agda{TXTree}{TXTree}

In this implementation, time is the number of transactions in TXTree.
Block is related to which block the transaction tree is.
After every new coinbase transaction (the miner transaction), the block size increment in one quantity.
Total fees are how much the miner will have in fee of transactions if he makes a block with these
transactions.
Quantity of transactions is how many transactions there are in the current block.
The type is tQtTxs instead of a natural number because, in this implementation, each block can have
a number maximum of transactions.
In Bitcoin, it is different, each block has a limit size in space of 1 MB.

Genesis tree is the first case.
It is when the cryptocurrency was created.
\emph{txtree} is created from another tree.
\emph{proofLessQtTX} is proof that the last transaction tree has its
block size less than the maximum block size minus one or it is a coinbase transaction.
It is because it is necessary to verify the size of the last \emph{txtree} so it will not have
the size greater than the maximum.

\agda{TXTree}{TX}

TX is related to the transaction done in the cryptocurrency.
There are two kinds of transaction.
Coinbase transaction is the transaction done by the miner.
In coinbase, they have just outputs and do not have any input.
\emph{pAmountFee} is proof that the output of the coinbase transaction is equal to the total fees plus
a block reward.

Another kind of transaction is the \emph{normalTX}, a regular transaction.
\emph{SubInputs} are a sub-list of all unspent transaction outputs of the previous transaction tree.
Outputs are the new unspent transaction from this transaction.
So who receives the amount from this transaction can spend it after.
\emph{TxSigned} is the signature that proves that every owner of each input approve this transaction.
In \emph{TxSigned}, there is proof that the output amount is greater than the input amount too.

\phantomsection
\label{iscoinbase}
\agda{TXTree}{isCoinBase}

\hyperref[iscoinbase]{This function} just returns trivial type if coinbase and bot type if not.

\agda{TXTree}{nextBlock}

If it is a normal transaction, the block continues the same.
If it is a coinbase transaction, the next transaction will be in a new block.

\phantomsection
\label{incqttx}
\agda{TXTree}{incQtTx}

\hyperref[incqttx]{This function} is to increment the number of transactions in the block.
It has to receive proof that the quantity of transaction that was before this new transaction was
less than then the maximum quantity of transactions allowed.
So it is guaranteed that the number of transactions will never be greater than the maximum allowed.
If it is a coinbase transaction, it will be a new block.
So the number of transactions starts being zero.

\phantomsection
\label{incfees}
\agda{TXTree}{incFees}

\hyperref[incfees]{IncFees} is a function that increments how much fee the miner will receive.
If it is a coinbase transaction, the fee will be received by the miner,
so the next miner will not receive this previous fee.
Because of that, the new fee will start from zero.
If it is a normal transaction, the newest fee will be the amount of input of the transaction minus
the output of this transaction plus the last fee of previous transactions.

\agda{TXTree}{outFee}

\emph{outFee+RewardBlock} is proof that the amount of output transactions is equal to total fees of
other transactions plus the block reward.

\subsubsection{Raw Transaction Tree}

The raw transaction tree is the tree without the explicit types.
Here, the definition:

\agda{RawTXTree}{rawtxtree}

A good utility of raw data types is that it is not necessary to add type arguments in functions.
Here, a function that adds a transaction to a transaction tree.
If this transaction is compatible with the transaction tree,
it returns a new transaction tree.
If it is not compatible, it returns nothing.
A better solution is a proof that this transaction is invalid with the transaction tree
instead of nothing.
But defining what is an invalid transaction can be tricky.

\agda{RawTXTree}{addtxtree1}

There are two cases.
The first one is if the transaction is a coinbase transaction.
It tries first to transform a list of \emph{TXField} into \emph{VecOut}.
If it can not transforms, it returns nothing.
If it can, it validates if the amount of vector output is equal to total fees plus the block reward.
After, it validates if the time of the transaction is equal to the time of the transaction tree.
In the end, it adds the outputs of the transaction to the vector of outputs.
Because it is a coinbase transaction, there are no inputs to be removed.

\agda{RawTXTree}{addtxtree2}

The second case, that the transaction is regular, looks like the same.
First, it validates if the quantity of transaction is less than the maximum allowed.
Second, it validates if this transactions is a valid signed transaction.
If all these conditions are true, it returns a new transaction tree
with news outputs equal to the outputs of this transaction plus the outputs of the last transaction tree
minus the inputs.
In case of an invalid transaction, it returns nothing.

\subsubsection{Proofs}

One of the important proofs is that each output of \emph{outputs transaction} is distinct.
This is very important because it guarantees that each input in the transaction could be
just related to just one unspent output.
This characteristic could be in the type of transaction tree,
but it is proven outside of it.

First, it is necessary to define what is a distinct union:

\agda{Utils}{uniondist}

The union of distinct lists makes a new distinct list if both are distinct to each other.

Now, to prove that outputs are a distinct list:

\agda{proofsTXTree}{uniqueouts}

In the first case, the transaction tree is a genesis tree without any outputs.
So an empty list is a distinct list.
In the second case, the outputs are the union of inputs of the transaction with the outputs
of \emph{vector output}.
So, it is necessary to prove that inputs of the transaction are distinct,
that elements of \emph{vector output} are also distinct and that both lists are distinct to each other.

\agda{proofsTXTree}{distinps}

There are some cases to prove that inputs are distinct.
First, if it is a regular transaction or if it is a coinbase transaction.
Second, if the transaction tree of this transaction is a genesis tree or if it is a regular tree.

If the transaction tree of the transaction is a genesis tree,
the number of inputs is zero.
So they are distinct.

In other cases, it does the same thing as proof of unique outputs.
The only difference is that it also does a recursive proof.
It assumes that the transaction of the last transaction tree is also distinct.

\agda{proofsTXTree}{alldists}

Both are distinct to each other because all of the transactions of input has the timeless
then the time of the transaction.
And because all of the outputs of the current transaction has time equal to the current time
of this transaction.

\agda{proofsTXTree}{outtimeless}

The proof that the time of the outputs is less than the current time of the transaction
is done recursively.
It is both necessary to proof that \emph{inputs of tx} and \emph{vector output} have both
times less than the current time of this transaction.
It is all done recursively.
