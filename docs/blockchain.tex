\section{Blockchain}

\subsection{Definition}

Block is a chain of transactions that is added in Bitcoin blockchain in every ten minutes.
Each block consists of several transactions and a miner transaction.
This is how a block is defined in this work:

\agda{Blockchain}{block}

\emph{nextTXTree} assures that the second transaction tree is from the first transaction tree.
\emph{firstTreesInBlock} guarantees that the last transaction in the first transaction tree
is the first in the block.
\emph{coinBaseTree} assures that the last transaction in the second transaction tree is
a coinbase transaction.

Blockchain is a chain of valid blocks.
Every new block must be a continuation of the previous one.
Here is the definition of the blockchain:

\agda{Blockchain}{blockchain}

In the first case, blockchain just has one block, called \emph{fstBlock}.
In the second case, the blockchain is an addition of a valid block from a previous blockchain.

\subsection{Creation}

To create a blockchain, it is first needed to create the last block.
From the last block, it is possible to create all the chain.

\agda{Blockchain}{blockblockchain}

In this proof, if the first transaction tree of the block is a genesis tree,
it will return a blockchain of just one block.
If it is a regular tree, it tries to find the first transaction tree of this block.
Using a recursive definition of block to blockchain,
it is possible to generate all the rest of this blockchain from this block.

It is not always possible to generate a block from the transaction tree.
It is because the last transaction of a transaction tree must be a coinbase transaction.
Here, the function that returns a decidable if it is possible to generate a block from
the transaction tree.

\agda{Blockchain}{treeblock}

The definition of the raw block gets just the coinbase transaction tree as an explicit type.
The other transaction tree can be founded opening the record.

\agda{Blockchain}{rawblock}

The code of the definition of what is a coinbase tree:

\agda{Blockchain}{coinbasetree}

The definition of a coinbase tree is the one that the last transaction is a coinbase.

The code verifies if the last transaction tree is a coinbase tree:

\agda{Blockchain}{iscoinbase}

If it is, it returns that it is possible to create a block from that with the block definition.
If it is not, it returns that it is impossible to create a block from this transaction tree.

But to create a block from this coinbase transaction tree, it is necessary to find the first tree
of the block.

\agda{Blockchain}{fsttree}

The definition of \emph{fstTree} is that it has a tree that is before this tree in the type.
And this tree before is the first in the block.

\agda{Blockchain}{firsttreeinblock}

The decidable version of this \emph{Set}:

\agda{Blockchain}{isfirsttreeinblock}

In this case, it pattern match trees that are genesis tree or if the last transaction was a coinbase
transaction.

\agda{Blockchain}{firsttree}

To find the first tree in the block, there are two cases.
The first case is that if the tree is a genesis tree, so the result is itself.
The second case is if it a regular tree, so it still has to divide it in many cases.
If this tree is already the first tree in the block, it will return itself.
If this tree is not, it has to verify if the block number of the tree is the same as this tree.
If the block number is equal, it can recursively find the first tree.
If it is not, it has to provide proof that this tree must be
the first and the blocks numbers are different.

To define what it means of one tree is next to another:

\agda{Blockchain}{nexttree}

There are two cases. If both trees are the same, they are next to each other.
If there is a proof that both trees are next to each other and
if there is one tree that was generated from the last one,
so the first tree is next to the last one.
